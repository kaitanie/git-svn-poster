\documentclass[final,hyperref={pdfpagelabels=false},notitlepage=true]{beamer} 
\usepackage{times}
\usepackage{listings}
\usepackage{amsmath,amssymb}
\usepackage[english]{babel}
\usepackage[latin1]{inputenc}
\usepackage[orientation=portrait,size=a0,scale=1.4,debug]{beamerposterbuiltin}   % e.g. for DIN-A0 poster
%\usepackage[size=custom,width=200,height=120,scale=2,debug]{beamerposter}  % e.g. custom size poster
% ...
\usepackage{xspace}
\usepackage{fp}
\usepackage{ifthen}

\useoutertheme{default}
\useoutertheme{hipinfolines}
\useinnertheme{rounded}

\title[]{\Huge Using Git Distributed Version Control Tool to Manage a HEP Data Analysis Project}

\author{A. Heikkinen\inst{1}, \underline{P. Kaitaniemi\inst{1,2}}}

\institute[] % (optional, but mostly needed)
{
  \inst{1}%
  Helsinki Institute of Physics P.O.Box 64 (Gustaf H\"allstr\"omin katu 2), FIN-00014 University of Helsinki, Finland
  \\
  \inst{2}%
  CEN-Saclay, CEA-IRFU/SPhN, 91 191 Gif sur Yvette, France
}

\date[March 12th, 2009]{March 12th, 2009}

%Add imafes/hiplogo.png here and CEN-CEA logo as well

\begin{document}
  \begin{frame}{} 
    \begin{center}
    \maketitle
    \end{center}
    \vfill
    \begin{abstract}
Version control allows software developers to keep track of project history in
a systematic and detailed manner. 
%The project can be a data analysis or simulation program source code, 
%or even LaTeX code of a paper or thesis. 
In addition to tracking history, version control tools allow
one to merge contributions between several authors.
This is a typical use-case for large projects in high energy physics.


% background
\vspace{2cm}
Recently \emph{CERN has chosen to upgrade centralised CVS version control
system \cite{cernsvn} to another modernised centralised system called Subversion (SVN)} \cite{svnsite}.
Unfortunately it does not offer the flexibility of the
distributed systems we are interested in. 
Git, a distributed version control tool \cite{torvalds}, however, 
provides this flexibility, 
and can be used as a ``super client'' with the CERN SVN service.

%We discuss advantages of distributed version control tools,
%such as Git, over traditional centralised ones. 

% our case and example image

%We present an example use case for Git in a HEP data analysis and publication writing project (Fig.~1)~\cite{pk09aProceedings}.
%We also demonstrate how to use Git
%together with CERN central SVN version control for high energy physics data
%analysis software maintenance.

    \end{abstract}
    \begin{columns}[t]
      \begin{column}{0.45\linewidth}

    \begin{block}{\large What is version control?}
      Benefits of version control for a physics software project.
      \begin{itemize}
        \item Keep detailed record of all changes to the project.
        \item Merge contributions from several authors and track author information
%        \item 
      \end{itemize}

\vpace{1cm}
      Generally version control works best for \emph{text files}, such
      as code or \LaTeX documents and
most significant advantage of distributed version control is that 
users can utilise it locally without heavy support infrastructure.

    \end{block}

\vpace{1cm}
    \begin{block}{\large Distributed version control}
      Distributed version control means that there is \emph{no single server}
      that keeps track of all data. This has several benefits:
      \begin{itemize}
        \item No need for special server machine
        \item Easy to set up:
          \begin{enumerate}
            \item {\tt mkdir myproject} \&\& {\tt cd myproject}
            \item {\tt git init}
          \end{enumerate}
        \item Works locally on users own machine. Network is needed
          only for sharing changes with collaborators.
        \item High performance: local operations are fast!
        \item Built for branching and merging:
          \begin{itemize}
            \item {\bf Branching:} Create additional line of
              development for a new idea.
            \item {\bf Merging:} Combine separate lines of development
              together.
          \end{itemize}
      \end{itemize}
    \end{block}

    \begin{block}{\large Version history}
      \includegraphics[scale=1.0]{images/gitk-history.png}
    \end{block}

    \begin{block}{\large Git GUI, graphical commit tool}
      \includegraphics[scale=1.0]{images/gui-screenshot.png}
    \end{block}

    \begin{block}{\large Git GUI Blame}
      Find out who is responsible for each line of code/text. \emph{Unique
      feature of Git blame is that it can heuristically find similar
      content in other files} that are part of the project thus
      allowing us to track content history beyond file boundaries.
      \includegraphics[scale=1.0]{images/git-gui-blame-content-copy-detection.png}
    \end{block}

    \begin{block}{\large Subversion at CERN}
      CERN has chosen Subversion revision control to replace its aging CVS based system.
    \end{block}

    \end{column}
      \begin{column}{0.45\linewidth}

	\begin{block}{\large Git-SVN: bidirectional Git-SVN gateway}
	  It is possible to fetch project revisions from SVN repository into a Git 
          repository, work locally and when finished the local commits
          can be pushed back to the central SVN repository.

          Example:
          \begin{itemize}
            \item {\bf Clone from SVN:} {\tt mkdir root-svn \&\& cd root-svn \&\& git svn init https://svn.root.cern.ch/trunk}
            \item {\bf Fetch all revisions:} {\tt git svn fetch}
            \item Work locally creating a series of commits
            \item {\bf Merge with the central repository:}
              \begin{itemize}
                \item {\bf Fetch new revisions:} {\tt git svn fetch}
                \item {\bf Apply your commits on top of the central revisions:} {\tt git svn rebase}
              \end{itemize}
            \item {\bf Push your changes to the central repository:} {\tt git svn dcommit}
          \end{itemize}

          Git-SVN can also be used as an SVN-to-Git conversion tool
          for groups that decide to completely switch to Git.
	\end{block}

        \begin{block}{\large Workflows}
          \includegraphics[scale=1.00]{images/centralizedWorkflow.pdf}
          \includegraphics[scale=1.00]{images/collaborativeDevelopmentWithMaintainer.pdf}
        \end{block}
%    \begin{block}{\large Fontsizes}
%      \centering
%      {\tiny tiny}\par
%      {\scriptsize scriptsize}\par
%      {\footnotesize footnotesize}\par
%      {\normalsize normalsize}\par
%      {\large large}\par
%      {\Large Large}\par
%      {\LARGE LARGE}\par
%      {\veryHuge veryHuge}\par
%      {\VeryHuge VeryHuge}\par
%      {\VERYHuge VERYHuge}\par
%    \end{block}

\begin{block}{\large Bibliography}
%%%
%%% The bibliography: the references are listed here.
%%%
\begin{thebibliography}{9}
\bibitem{cernsvn}
\href{http://cern.ch/svn}{CERN central SVN service (http://cern.ch/svn)}

\bibitem{svnsite}
\href{http://subversion.tigris.org}{Subversion website (http://subversion.tigris.org)}

\bibitem{torvalds}
L.Torvalds with the Linux kernel team,
Git Version Control System website,
\href{http://git-scm.com/}{http://git-scm.com/}

%\bibitem{gitsite} % same as \bibitemtorvalds
%\href{http://git.or.cz}{Git website (http://git.or.cz)}%

\bibitem{pk09aProceedings}
P.~Kaitaniemiemi and A.~Heikkinen et al.,
Ideal $\tau$-tagging with TMVA multivariate data-analysis toolkit,
Proceedings of International Conference on 
Computing in High Energy and Nuclear Physics, CHEP'09
(To be published)

\end{thebibliography}
\end{block}
    \vfill
    \end{column}
    \end{columns}
  \end{frame}
\end{document}
%%%%%%%%%%%%%%%%%%%%
%%% Local Variables: 
%%% mode: latex
%%% TeX-PDF-mode: t
%%% End: 
